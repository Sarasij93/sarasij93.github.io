%% LyX 2.1.4 created this file.  For more info, see http://www.lyx.org/.
%% Do not edit unless you really know what you are doing.
\documentclass[english, leqno]{scrartcl}
\usepackage[T1]{fontenc}
\usepackage[utf8]{luainputenc}
\usepackage{color}
\usepackage{float}
\usepackage{fancybox}
\usepackage{calc}
\usepackage{amsmath}
\usepackage{amsthm}
\usepackage{esint}
\oddsidemargin=-.5in
\evensidemargin=-.5in
\textwidth=7in
\textheight=9.5in
\makeatletter

%%%%%%%%%%%%%%%%%%%%%%%%%%%%%% LyX specific LaTeX commands.
%% Because html converters don't know tabularnewline
\providecommand{\tabularnewline}{\\}

\@ifundefined{date}{}{\date{}}
%%%%%%%%%%%%%%%%%%%%%%%%%%%%%% User specified LaTeX commands.
\usepackage{tikz}
\usepackage{pgfplots}
\usetikzlibrary{matrix,arrows,decorations.pathmorphing}
\usepackage{verbatim}
\usepackage{pgf}
\usepackage{color}
\definecolor{BLUE}{RGB}{0, 128, 255}
\definecolor{Green}{RGB}{30, 200, 140}
\definecolor{green}{RGB}{0, 200, 140}
\definecolor{GREEN}{RGB}{0, 200, 140}

\renewcommand\[{\begin{equation}}
\renewcommand\]{\end{equation}}
\usetikzlibrary[patterns]

\makeatother

\usepackage{babel}
\begin{document}

\title{\textcolor{blue}{Official Study Guide Final Exam}}

\maketitle
\shadowbox{\begin{minipage}[t]{1\columnwidth}%
\textbf{\textcolor{blue}{Information for Final Exam:}}
\begin{itemize}
\item Tuesday December 12th
\item 7:00 pm - 10:00 pm 
\item Location: Instructors add your exam location
\item Topics on the Exam: \textcolor{red}{The Final Exam will be cumulative,
at most 30 out of 100 points will be material covered since Exam 2,
that is, sections 6.1-6.5. To prepare for the final, you
should review topics covered on Exams 1 and 2, assisted by the study
guides for those exams. To review material covered since Exam 2, use
this guide. The study guides should not be your only material used
when studying for the final exam. Reviewing previous WebAssign sets,
midterm exams, quizzes, written homework assignments, the textbook, and your notes
is all recommended, as is completing posted practice final exams.}
\smallskip

\textcolor{red}{
Note: For all limit problems, your work must be well-organized.  In particular, for a limit that exists, a string of valid equalities must connect the original limit problem with the value of the limit.  For limits at infinity, you won't be required to justify your answer: giving the correct limit will be considered sufficient. 
}\end{itemize}
%
\end{minipage}}



\begin{center}
\textbf{\textcolor{cyan}{\large{}The following list of topics is not
exhaustive, it is just meant to highlight the most important aspects
of the chapter}}
\par\end{center}{\large \par}



\subsection*{\textcolor{blue}{Topics to Keep in Mind from Chapter 6:}}
\begin{itemize}
\item If $f(x)$ is a function, all possible antiderivatives of $f(x)$
on an interval of interest are represented by the symbol $\int f(x)dx$.
Under this notation, we might say that 
\[
{\color{blue}\int f(x)dx= \textrm{ the family of all functions having derivative }f(x)};
\]
We might also say, 
\[
{\color{blue}\int f(x)dx=\textrm{the general antiderivative of } f(x)}.
\]
For example, $\int xdx=\frac{x^{2}}{2}+C$ means that all functions
having derivative   $f(x) = x$ on say, $(-\infty,\infty)$ are of the form
$\frac{x^{2}}{2}+C$ where $C$ is an arbitrary constant

\item Remark:  In general, finding antiderivatives is not as straightforward as finding derivatives. There are simple-looking functions such as $f(x) = e^{x^2}$ that do not have ``nice antiderivatives''.\footnote{ Roughly speaking, ``not nice'' here means ``can't be expressed as a finite algebraic combination of functions with which you are familiar''.}  On the final exam, if you are asked to find an antiderivative of a function $f$ or to evaluate an indefinite integral $\int f\, dx$, then you will be dealing with a function $f$ that {\it does} have a nice antiderivative, which you'll be able to find using the integration/antidifferentiation rules and techniques you have learned in this course.   

\item You should know the following integration rules.  
\begin{center}
\begin{tabular}{|c|c|}
\hline 
Rule & Notation\tabularnewline
\hline 
\hline 
Integral of a constant & ${\color{blue}\int kdx=kx+C}$\tabularnewline
\hline 
Constants slide through  integrals & ${\color{blue}\int kf(x)dx=k\int f(x)dx}$\tabularnewline
\hline 
The integral of a sum is the sum of the integrals & ${\color{blue}\int\left(f(x)+g(x)\right)dx=\int f(x)dx+\int g(x)dx}$\tabularnewline
\hline 
Integral of a power of $x$ & ${\color{blue}\int x^{n}dx=\frac{1}{n+1}x^{n+1}+C\qquad n\neq-1}$\tabularnewline
\hline 
Integral of $\frac{1}{x}$ & ${\color{blue}\int\frac{dx}{x}=\ln\left|x\right|+C}$\tabularnewline
\hline 
Integral of an exponential & ${\color{blue}\int e^{x}dx=e^{x}+C}$\tabularnewline
\hline 
\end{tabular}
\par\end{center}

\item {\it Your work on indefinite integrals must be well organized; in particular, you must have a string of valid  equalities connecting the original indefinite integral with your final answer.}

\item An \textbf{\textcolor{blue}{Initial Value Problem}} (IVP) is a problem
of the form 
\[
{\color{blue}\begin{cases}
\frac{dy}{dx}=g(x)\\
y\left(x_{0}\right)=y_{0}
\end{cases}}
\]
That is, you want to find a function $y(x)$ whose derivative is $g(x)$
and whose value at $x_{0}$ is $y_{0}$. To solve an IVP do the following:
\begin{enumerate}
\item Integrate both sides of the differential equation $ \frac{dy}{dx}=g(x)$, obtaining
$$
y(x)=G(x)+C,
$$
 where $G$ is an antiderivative of
$g$ and $C$ is a constaht
\item  We have $
y(x)=G(x)+C,
$
and upon substituting $x_0$ for $x$ in this equation, we obtain $y_0 = G(x_0) + C$, or  $C = y_0 - G(x_0)$.  The final solution is $y(x) = G(x) + C$, where $C =   y_0 - G(x_0)$. 
\end{enumerate}

For example, to solve the IVP 
\[
\begin{cases}
\frac{dy}{dx}=x^{2}-1\\
y(1)=-1
\end{cases}
\]
We use the first step to write $y(x)=\int\left(x^{2}-1\right)dx=\frac{x^{3}}{3}-x+C$ and then
choose the constant $C$ such that  $y(1)=-1$, that is, $\frac{1}{3}-1+C=-1$
which gives $C=-\frac{1}{3}$ so $y(x)=\frac{x^{3}}{3}-x-\frac{1}{3}$
is the solution of the IVP.
\smallskip

 As a physical application of initial value problems, we have that
the velocity of an object as a function of time $v(t)$ is an antiderivative
of its acceleration $a(t)$ and the position of an object as a function
of time $s(t)$ is an antiderivative of the velocity $v(t)$, that
is, 
$$
v(t)=\int a(t)dt \quad  \text{and}\quad  s(t)=\int v(t)dt, 
$$
where we interpret the preceding equations to specify the forms of $v$ and $s$, with constants of integration being determined by appropriate initial conditions. 
\smallskip

For example if the acceleration as a function of time is $a(t)=t^{2}-7$
then its velocity is $v(t)=\int a(t)dt=\int\left(t^{2}-7\right)dt=\frac{t^{3}}{3}-7t+C$
where the constant $C$ is determined once an initial condition is
given for $v$. 

\item You should be prepared to find indefinite integrals via the method of substitution, often called ``$u$-substitution.''  See Section 6.2 of the text.

\begin{itemize}
\item For example, to find the integral $\int\frac{\sqrt{\ln x}}{x}dx$
we make the substitution $u=\ln x$. In this case $du=\frac{dx}{x}$
and the integral $\int\frac{\sqrt{\ln x}}{x}dx$ becomes 
\[
\int\frac{\sqrt{\ln x}}{x}dx=\int\sqrt{u}\, du=\frac{2}{3}u^{\frac{3}{2}}+C=\frac{2}{3}\left(\ln x\right)^{\frac{3}{2}}+C
\]
\end{itemize}

\item You should know how definite integrals are defined via {\it Riemann Sums}.  

{\small{}}%
\doublebox{\begin{minipage}[t]{1\columnwidth}%
\textbf{\textcolor{blue}{\small{}Definition of a Riemann Sum and the
Definite Integral:}}{\small{} Let $f$ be defined on $[a,b]$. The
definite integral of $f$ from $a$ to $b$, denoted $\int_{a}^{b}f(x)dx$
is given by
\[
{\color{blue}\begin{array}{ccc}
{\color{blue}\int_{a}^{b}f(x)} & {\color{blue}=} & {\color{blue}\displaystyle \lim_{n\to\infty}\left[f\left(x_{1}\right)\triangle x+f(x_{2})\triangle x+\cdots+f(x_{n})\triangle x\right]}\\
{\color{blue}} & {\color{blue}} & {\color{blue}}\\
{\color{blue}} & {\color{blue}=} & {\color{blue}\displaystyle \lim_{n\to \infty}\left[f\left(x_{1}\right)+f(x_{2})+\cdots+f(x_{n})\right]\left(\frac{b-a}{n}\right)}
\end{array}}
\]
 provided the limit exists and takes the same value independent of
how the numbers $x_{i}\text{ in }\left[a+(i-1)\frac{(b-a)}{n},a+i\frac{(b-a)}{n}\right]$
are chosen. Tan calls these numbers $x_{1},x_{2},\cdots,x_{n}$ representative
points. }{\small \par}
\begin{itemize}
\item When the limit defining $\int_{a}^{b}f(x)dx$ exists we say that $f$
is integrable on $[a,b]$. It is possible to show that if $f$ is
continuous on $[a,b]$ then $f$ is integrable on $[a,b]$.
\item Because continuous functions are integrable, we can simplify our definition
of definite integral if the integrand $f$ is continuous as follows:\end{itemize}
\begin{enumerate}
\item \textbf{\textcolor{blue}{\small{}Evaluation via left-endpoint approximation:
}}\textcolor{black}{assume $f$ is continuous on $[a,b]$, then choosing
representative points as left-endpoints, we have}{\small{} 
\[
{\color{blue}\int_{a}^{b}f(x)dx\equiv\lim_{n\longrightarrow\infty}\left(\frac{b-a}{n}\right)\left[f\left(a\right)+f\left(a+\frac{b-a}{n}\right)+\cdots+f\left(a+\left(n-1\right)\frac{\left(b-a\right)}{n}\right)\right]}
\]
For example,  Left Endpoint Approximations for $f(x)=x^{2}$ on
$[a,b]=[0,1]$ yield
\[
\int_{0}^{1}x^{2}dx=\lim_{n\longrightarrow\infty}\frac{1}{n}\left[\left(0\right)^{2}+\left(\frac{1}{n}\right)^{2}+\cdots+\left(\frac{n-1}{n}\right)^{2}\right]
\]
}{\small \par}
\item \textbf{\textcolor{blue}{\small{}Evaluation via right-endpoint approximation:
}}\textcolor{black}{assume $f$ is continuous on $[a,b]$, then choosing
representative points as right-endpoints, we have}{\small{} 
\[
{\color{blue}\int_{a}^{b}f(x)dx\equiv\lim_{n\longrightarrow\infty}\left(\frac{b-a}{n}\right)\left[f\left(a+\frac{(b-a)}{n}\right)+f\left(a+2\frac{\left(b-a\right)}{n}\right)+\cdots+f\left(b\right)\right]}
\]
For example,  Right Endpoint Approximations for $f(x)=x^{2}$ on
$[a,b]=[0,1]$ yield
\[
\int_{0}^{1}x^{2}dx=\lim_{n\longrightarrow\infty}\frac{1}{n}\left[\left(\frac{1}{n}\right)^{2}+\left(\frac{2}{n}\right)^{2}+\cdots+1^{2}\right]
\]
}{\small \par}\end{enumerate}
%
\end{minipage}}{\small \par}
\begin{itemize}
\item \textbf{\textcolor{red}{Remark 1): you should be able to compute the
Riemann sums for $f$ corresponding to left-endpoint approximations
and right-endpoint approximations.}} For example, if you are asked
to approximate $\int_{-1}^{2}(2x-x^{2}+1)dx$ using the left-endpoint
approximation using three subdivisions of the interval $\left[-1,2\right]$
you must calculate the following sum: (we are taking $[a,b]=[-1,2]$
, $n=3$ and $f(x)=2x-x^{2}+1$ in the formula for the left-endpoint
approximation)
\[
\left(\frac{2-(-1)}{3}\right)\left[f\left(-1\right)+f(0)+f(1)\right]=f(-1)+f(0)+f(1)=-2+1+2=1
\]

\end{itemize}
\begin{center}
\definecolor{ffxfqq}{rgb}{1.,0.4980392156862745,0.}
\definecolor{zzttqq}{rgb}{0.6,0.2,0.}
\definecolor{ffqqqq}{rgb}{1.,0.,0.}
\definecolor{qqwuqq}{rgb}{0.,0.39215686274509803,0.}
\begin{tikzpicture}[line cap=round,line join=round,>=triangle 45,x=1.5cm,y=1.5cm]
\draw[->,color=black] (-1.5,0.) -- (3.,0.);
\foreach \x in {-1.,1.,2.}
\draw[shift={(\x,0)},color=black] (0pt,2pt) -- (0pt,-2pt) node[below] {\footnotesize $\x$};
\draw[->,color=black] (0.,-2.5) -- (0.,2.5);
\foreach \y in {-2.,-1.,1.,2.}
\draw[shift={(0,\y)},color=black] (2pt,0pt) -- (-2pt,0pt) node[left] {\footnotesize $\y$};
\draw[color=black] (0pt,-10pt) node[right] {\footnotesize $0$};
\clip(-1.5,-2.5) rectangle (3.,2.5);
\fill[color=zzttqq,fill=zzttqq,fill opacity=0.1] (-1.,0.) -- (-1.,-2.) -- (0.,-2.) -- (0.,0.) -- cycle;
\fill[color=zzttqq,fill=zzttqq,fill opacity=0.1] (0.,1.) -- (0.,0.) -- (1.,0.) -- (1.,1.) -- cycle;
\fill[color=zzttqq,fill=zzttqq,fill opacity=0.1] (1.,2.) -- (1.,0.) -- (2.,0.) -- (2.,2.) -- cycle;
\draw[line width=1.2pt,color=qqwuqq,smooth,samples=100,domain=-1.5:3.0] plot(\x,{2.0*(\x)-(\x)^(2.0)+1.0});
\draw [color=ffxfqq] (-1.,0.)-- (-1.,-2.);
\draw [color=ffxfqq] (-1.,-2.)-- (0.,-2.);
\draw [color=ffxfqq] (0.,-2.)-- (0.,0.);
\draw [color=ffxfqq] (0.,0.)-- (-1.,0.);
\draw [color=ffxfqq] (0.,1.)-- (0.,0.);
\draw [color=ffxfqq] (0.,0.)-- (1.,0.);
\draw [color=ffxfqq] (1.,0.)-- (1.,1.);
\draw [color=ffxfqq] (1.,1.)-- (0.,1.);
\draw [color=ffxfqq] (1.,2.)-- (1.,0.);
\draw [color=ffxfqq] (1.,0.)-- (2.,0.);
\draw [color=ffxfqq] (2.,0.)-- (2.,2.);
\draw [color=ffxfqq] (2.,2.)-- (1.,2.);
\begin{scriptsize}
\draw [fill=ffqqqq] (-1.,-2.) circle (2.5pt);
\draw [fill=ffqqqq] (0.,1.) circle (2.5pt);
\draw [fill=ffqqqq] (1.,2.) circle (2.5pt);
\end{scriptsize}
\end{tikzpicture}
\par\end{center}

\ 

\textbf{\textcolor{red}{See also problems 3,4,14 and 15 on pages 428
and 429.}}
\begin{itemize}
\item \textbf{\textcolor{red}{Remark 2): you should be able to compare left-endpoint
(or right-endpoint) Riemann sums for $f$ on $[a,b]$ to the value
of $\int_{a}^{b}f(x)dx$ in certain situations.}} For instance, when
$f$ is positive and increasing on $[a,b]$, you should know that
finite left endpoint approximation is an underestimate of $\int_{a}^{b}f(x)\,dx$
and a finite right endpoint approximation is an overestimate of$\int_{a}^{b}f(x)\,dx$.
When $f$ is positive and decreasing on$[a,b]$, then a finite left
endpoint approximation is an overestimate of$\int_{a}^{b}f(x)\,dx$
and a finite right endpoint approximation is an underestimate of $\int_{a}^{b}f(x)\,dx$.
\end{itemize}
\item You should know the following properties of the definite integral.
\smallskip

 \textbf{\textcolor{blue}{The Definite Integral and Area:}} $\int_{a}^{b}f(x)dx$
is the\textbf{ }\textbf{\textcolor{blue}{signed or net area}} under/over
the curve of $y=f(x)$. Area should always be deemed positive. If
the curve is above the $x$ axis, then integrating left to right yields
the area under the curve and above the $x$ axis, whereas if the curve
is below the $x$ axis, the integrating left-to-right yields negative
the area above the curve and below the $x$ axis.  In general, assuming $a < b$, the integral $\int_{a}^{b}f(x)dx$ is a difference of areas: the area under the curve $y = f(x)$ and  above the interval $[a,b]$ minus the area above the curve and under the interval $[a,b]$.  See the picture on page 427 of the text.
\smallskip


{\small
 \doublebox{\begin{minipage}[t]{1\columnwidth}%
\textbf{\textcolor{blue}{Additional Properties of the definite integral}}: Suppose
that $f(x),g(x)$ are functions defined on $[a,b]$ for which $\int_{a}^{b}f(x)dx$,
$\int_{a}^{b}g(x)dx$ exist. If $c$ is any number between $a$ and
$b$ and $k$ any constant then 
\begin{itemize}
\item $\int_{a}^{a}f(x)dx=0$
\item $\int_{a}^{b}f(x)dx=\int_{a}^{c}f(x)dx+\int_{c}^{b}f(x)dx$
\item $\int_{a}^{b}\left(f(x)+kg(x)\right)dx=\int_{a}^{b}f(x)dx+k\int_{a}^{b}g(x)dx$
\item $\int_{a}^{b}f(x)dx=-\int_{b}^{a}f(x)dx$ 
\item The last property allows us to split the integral $\int_{c_{1}}^{c_{2}}f(x)dx$
as 
\[
{\color{blue}\int_{c_{1}}^{c_{2}}f(x)dx=\int_{c_{1}}^{c_{3}}f(x)dx+\int_{c_{3}}^{c_{2}}f(x)dx}
\]
\textcolor{blue}{regardless of the order of the points $c_{1},c_{2,}c_{3}$}!
For example, $\int_{1}^{-2}x^{2}dx=\int_{1}^{0}x^{2}dx+\int_{0}^{-2}x^{2}dx$. \end{itemize}
%
\end{minipage}}

}



\item  \textbf{\textcolor{red}{You must be able to state the Fundamental
Theorem of Calculus}}:

\shadowbox{\begin{minipage}[b]{1\columnwidth}%
 Let $f$ be a continuous function on $[a,b]$.
Then $\int_{a}^{b}f(x)dx=F(b)-F(a)$ where $F$ is any antiderivative
of $f$; that is, $F'(x)=f(x)$.%
\end{minipage}}
\begin{itemize}
\item When computing definite integrals via the Fundamental Theorem, sometimes you will use
the method of substitution as part of the process. There are two natural ways to organize your work (see pages 443 and 444 of the textbook), which we will now illustrate.  {\it  Note well:  you must organize your work in one of these ways. In particular, the original definite integral must be connected by a string of valid equalities (showing how the Fundamental Theorem of Calculus is being applied)  to your final answer }


 For example,
suppose that we want to find $\int_{0}^{\sqrt{2}}xe^{x^{2}}dx$. 
\item FIRST METHOD: you compute first the corresponding indefinite integral
$\int xe^{x^{2}}dx$, for which we use the substitution $u=x^{2}$,
$du=2xdx$ or $\frac{1}{2}\, du = x\, dx$.  We have
\[
\int xe^{x^{2}}dx= \frac{1}{2} \int e^{u}du=\frac{1}{2}e^{u}+C=\frac{1}{2} e^{x^{2}}+C.
\]
Thus, $F(x) = \frac{1}{2}e^{x^{2}}$ is an antiderivative of $f(x) = xe^{x^{2}}$ and  the Fundamental
Theorem of Calculus allows us to conclude that 
\[
\int_{0}^{\sqrt{2}} xe^{x^{2}}dx=\left. \left(\frac{1}{2}e^{x^{2}}\right)\right|_{0}^{\sqrt{2}}=\left(\frac{1}{2}e^{2}-\frac{1}{2}e^{0}\right)=\frac{1}{2}e^{2}-\frac{1}{2}.
\]
\item SECOND METHOD: we choose to work with the definite integral from the
beginning, but now we need to keep track of the limits of integration.
In order to find $\int_{0}^{\sqrt{2}} xe^{x^{2}}dx$ we use again
the substitution $u=x^{2}$ , $du=2xdx$ but now we keep track of
the limits of integration. Since 
\[
\begin{cases}
x=0\implies u=0\\
x=\sqrt{2}\implies u=2
\end{cases}
\]
the definite integral becomes 
\[
\int_{0}^{\sqrt{2}}xe^{x^{2}}dx= \frac{1}{2}\int_{0}^{2}e^{u}du=\frac{1}{2}\left.\left(e^{u}\rule{0in}{0.16in}\right)\right|_{0}^{2}=\frac{1}{2}\left(e^{2}-e^{0}\right)=\frac{1}{2}\left(e^{2}-1\right).
\]
The advantage of this method is that after you find the integral in
terms of $u$ there is no need to use the substitution again in order
to get an expression in terms of $x$ since the new limits of integration
take care of that. However, if you use this method be sure to change
the limits of integration!
\end{itemize}

\item You should know the Net-Change Formula:   The net change  in a function $f$ over an interval $[a,b]$ is given by
$$
f(b) - f(a) = \int_a^b f'(x)\, dx,
$$
provided that $f'$ is continuous in $[a,b]$.  
\item You should know the formula for the average
value.   Suppose that $f(x)$ is a function defined on $[a,b]$. Then its\textbf{\textcolor{blue}{{}
average value over $[a,b]$}} is by definition 
$$
{\color{blue}\frac{1}{b-a}\int_{a}^{b}f(x)dx}
$$

\end{itemize}

\end{document}


\doublebox{\begin{minipage}[t]{1\columnwidth}%
\textbf{\textcolor{blue}{Method of Substitution: }}
\begin{enumerate}
\item Let $u=g(x)$ where $g(x)$ is part of the integrand, usually the
``inside'' function of a composite function $f(g(x))$
\item Find $du=g'(x)dx$
\item Use the substitution $u=g(x)$ and $du=g'(x)dx$ to convert the entire
integral into one involving only $u$
\item Find the resulting integral
\item Replace $u$ by $g(x)$ to obtain the final solution as a function
of $x$\end{enumerate}
%
\end{minipage}}